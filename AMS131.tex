% LaTeX template adapted from: https://www.overleaf.com/latex/templates/simple-math-homework-template/tbszsswsndrz
\documentclass{article}
\usepackage[utf8]{inputenc}
\usepackage[english]{babel}
\usepackage[]{amsthm} %lets us use \begin{proof}
\usepackage[]{amssymb} %gives us the character \varnothing
\usepackage{amsmath} %for equations
\usepackage[]{listings} %for code blocks
\usepackage{graphicx} %for diagrams
\usepackage{fancyhdr} %for headers
\usepackage[letterpaper, margin=1in]{geometry}
\usepackage{tikz} % for drawings
\usepackage{multicol}
\usetikzlibrary{arrows.meta,shapes.arrows,chains,decorations.pathreplacing}


\graphicspath{}
\pagestyle{fancy}
\setlength{\parindent}{1em}
\setlength{\parskip}{0em}
\rhead{Pete Wilcox | pcwilcox@ucsc.edu}
\lhead{AMS 131: Cheat Sheet}  

\begin{document}
\begin{multicols}{3}
    \subsection*{Experiments and Events:}
        \paragraph*{Def:} 
            An experiment is a process whose outcome is not known in advance with certainy.

        \paragraph*{Sample Space: } 
            Collection of \textit{all} possible outcomes of an experiment. $S$ or $\Omega$. Each outcome is an element of the sample space $s \in S$.

    \subsection*{Operations: }
        \paragraph*{Union: } 
            \( x \in S: A \cup B = \{x \in A \text{ or } x \in B\}\)
            
            \begin{itemize}
                \item[] \(A \cup B = B \cup A \)
                \item[] \(A \cup A = A \)
                \item[] \(A \cup \emptyset = A \)
                \item[] \(A \cup S = S \)
                \item[] \(A \subset B \Rightarrow A \cup B = B\)
                \item[] \(A_1, A_2, \dots, A_n \Rightarrow A_1 \cup A_2 \cup \dots \cup A_n = \bigcup\limits_{i=1}^{i=n} A_i\)
                \item[] \(\bigcup\limits_{i=1}^{\infty} A_i \rightarrow \bigcup\limits_{i\in I}A_i \)
                \item[] \( (A \cup B) \cup C = A \cup (B \cup C) = A \cup B \cup C \)
            \end{itemize}

        \paragraph*{Intersections: } 
            \( A \cap B = \{ x \in A \text{ and } x \in B\} = AB \)            
            
            \begin{itemize}
                \item[] \(A \cap B = B \cap A \)
                \item[] \(A \cap A = A \)
                \item[] \(A \cap \emptyset = \emptyset \)
                \item[] \(A \cap S = A\)
                \item[] \(A \subset B \Rightarrow A \cap B\)
                \item[] \(\bigcap\limits_{i\in I}A_i = \bigcap\limits_{i=1}^{\infty}A_i\)
                \item[] \(\bigcap\limits_{i=1}^{n}A_i = A_1 \cap A_2 \cap \dots \cap A_n \)
                \item[] \( (A \cap B) \cap C = A \cap (B \cap C) = A \cap B \cap C\)
            \end{itemize}
        
        \paragraph*{Complements: } 
            \(A^c = \{x \in S: x \notin A\}\)
            
            \begin{itemize}
                \item[] \((A^c)^c = A\)
                \item[] \(\emptyset^c = S\)
                \item[] \(S^c = \emptyset\)
                \item[] \(A\cup A^c = S\)
                \item[] \(A \cap A^c = \emptyset\)
            \end{itemize}
    
    \subsection*{Dijoint Events: } 
        $A$ and $B$ are \textit{disjoint} or \textit{mutually exclusive} if $A$ and $B$ have no outcomes in common. This happens only if \(A \cap B = \emptyset \).
        
        \paragraph*{Def:} 
            A collection \(A_1, \dots, A_n\) is a collection of disjoint evens if and only if \(A_i \cap A_j = \emptyset, \forall i, j, i \neq j\)
            
            \begin{itemize}
                \item[] \(\left (\bigcup\limits_{i\in I} A_i \right)^c = \bigcap\limits_{i\in I}A_i^c \)
                \item[] \((A\cup B)^c = A^c \cap B^c \)
                \item[] \(x\in(A\cap B)^c \\ 
                \Rightarrow x \notin A \text{ and } x\notin B \\
                \Rightarrow x \in A^c \text{ and } x \in B^c \\
                \Rightarrow x \in A^c \cap B^c \)
            \end{itemize}

    \subsection*{Probabilities: } 
        Functions over $S$ that measure the likelihood of events.

            \begin{itemize}
                \item[] $\forall A: Pr(A) \geq 0$
                \item[] $Pr(S) = 1$
                \item[] For every \textit{infinite sequence} of \textit{disjoint} events \( A_1, A_2, \dots (A_i \in S): \\
                Pr\left(\bigcup\limits_{i=1}^{\infty}A_i\right) = \sum\limits_{i=1}^{\infty} Pr(A_i) \\
                Pr(A_1 \cup A_2 \cup \dots \cup A_n \cup \dots) \\
                = Pr(A_1) + Pr(A_2) + \dots + Pr(A_n) + \dots \)
                \item[]\(Pr(\emptyset) = 0\)
                \item[]\(Pr(\bigcup\limits_{i=1}^{n}A_i)=Pr(\bigcup\limits_{i=1}^{n}A_i + \bigcup\limits_{i=n+1}^{\infty}\emptyset) = \sum\limits_{i=1}^{n}Pr(A_i)\)
                \item[]\(Pr(A^c) = 1-Pr(A)\)
                \item[]\(A \subset B \Rightarrow Pr(A) \leq Pr(B)\)
                \item[]\(\forall A: 0 \leq Pr(A) \leq 1\)
                \item[]\(Pr(A\cup B) = Pr(A) + Pr(B) - Pr(A\cap B)\)
            \end{itemize}

    \subsection*{Finite Sample Spaces: } 
        \(S := \{s_1, s_2, \dots, s_n \}\)
        
        \paragraph*{} 
            To obtain a probability distribution over $S$ we need to specify \(Pr(s_i) = P_i, \forall i = 1, 2, \dots, n\), such that \(\sum\limits_{i=1}^{n}P_i = 1\)

        \paragraph*{Def: } 
            A sample space $S$ with $n$ outcomes $s_1, \dots, s_n$ is a \textit{simple sample space} if the probability assigned to each outcome is $\frac{1}{n}$. If $A$ contains $m$ outcomes then \(Pr(A) = \frac{m}{n}\).

    \subsection*{Counting Methods: }
        
        \paragraph*{Multiplication Rule: } 
            Suppose an experiment has $k$ parts \((k \geq 2)\) such that the $i^{th}$ part of the experiement has $n_i$ possible outcomes, $i = 1, \dots, k$, and that \textit{all possible outcomes can occur regardless of which outcomes have occurred in other parts}. The sample space $S$ will contain vectors of the form $(u_1, u_2, \dots, u_k)$. $u_i$ is one of the $n_i$ possible outcomes of part $i$. The total number of vectors is $n_1 \cdot n_2 \cdot \dots \cdot n_k$.
        
        \paragraph*{Permutations: } 
            Given an array of $n$ elements the first position can be filled with $n$ different elements, the second with $n-1$, and so on. $n \cdot (n-1) \cdot (n-2) \cdot \dots \cdot 1 = n!$

            \begin{itemize}
                \item[] \(P_{n,k} = \frac{n!}{(n-k)!}\)
                \item[] \(P_{n,n} = n!\)
            \end{itemize}

        \paragraph*{Combinations: }
            In general we can "combine" $n$ elements taking $k$ at a time in \(C_{n,k} = \frac{P_{n,k}}{k!} = \frac{n!}{(n-k)!k!} = {n \choose k} \). 

        \paragraph*{Multinomial Coefficeints: } Consider splitting $n$ elements into $k (k \geq 2)$ groups in a way such that group $j$ gets $n_j$ elements and $\sum\limits_{j = 1}^{k}n_j = n$. The $n_1$ elements in the first group can be selected in ${n choose n_1}$

        \end{multicols}

\end{document}