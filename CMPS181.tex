% LaTeX template adapted from: https://www.overleaf.com/latex/templates/simple-math-homework-template/tbszsswsndrz
\documentclass[landscape,8pt]{extarticle}
\usepackage[utf8]{inputenc}
\usepackage[english]{babel}
\usepackage{extsizes}
\usepackage[]{amsthm} %lets us use \begin{proof}
\usepackage[]{amssymb} %gives us the character \varnothing
\usepackage{amsmath} %for equations
\usepackage[]{listings} %for code blocks
\usepackage{graphicx} %for diagrams
\usepackage{fancyhdr} %for headers
\usepackage[letterpaper, margin=0.25in, headheight=10pt, headsep=0pt]{geometry}
\usepackage{tikz} % for drawings
\usepackage{multicol}
\usepackage{ifthen}
\usepackage{enumitem}
\usepackage{listings}
\usepackage{color}
\usepackage{adjustbox}

\usepackage{wrapfig}
\usepackage{booktabs}
\setlist{nolistsep}
\setlist[itemize]{leftmargin=0.25pc,itemsep=0em}
\usetikzlibrary{arrows.meta,shapes.arrows,chains,decorations.pathreplacing}
\makeatletter
\renewcommand{\section}{\@startsection{section}{1}{0mm}%
            {-1ex plus -.5ex minus -.2ex}%
            {0.5ex plus .2ex}%x
            {\normalfont\large\bfseries}}
\renewcommand{\subsection}{\@startsection{subsection}{2}{0mm}%
            {-1explus -.5ex minus -.2ex}%
            {0.5ex plus .2ex}%
            {\normalfont\normalsize\bfseries}}
\renewcommand{\subsubsection}{\@startsection{subsubsection}{3}{0mm}%
            {-1ex plus -.5ex minus -.2ex}%
            {1ex plus .2ex}%
            {\normalfont\small\bfseries}}
\makeatother
\definecolor{dkgreen}{rgb}{0,0.6,0}
\definecolor{gray}{rgb}{0.5,0.5,0.5}
\definecolor{mauve}{rgb}{0.58,0,0.82}

\lstdefinelanguage{JavaScript}{
  keywords={typeof, new, true, false, catch, function, return, null, catch, switch, var, if, in, while, do, else, case, break},
  keywordstyle=\color{blue}\bfseries,
  ndkeywords={class, export, boolean, throw, implements, import, this},
  ndkeywordstyle=\color{darkgray}\bfseries,
  identifierstyle=\color{black},
  sensitive=false,
  comment=[l]{//},
  morecomment=[s]{/*}{*/},
  commentstyle=\color{purple}\ttfamily,
  stringstyle=\color{red}\ttfamily,
  morestring=[b]',
  morestring=[b]"
}


\lstset{frame=single,
  language=c,
  aboveskip=3mm,
  belowskip=3mm,
  showstringspaces=false,
  columns=fixed,
  basicstyle={\small\ttfamily},
  numbers=none,
  breaklines=true,
  breakatwhitespace=true,
  tabsize=3
}
\newcommand{\code}{\lstinline}
\graphicspath{}
\pagestyle{empty}
\ifthenelse{\lengthtest { \paperwidth = 11in}}
{ \geometry{top=.3in,left=.25in,right=.25in,bottom=.25in} }
{\ifthenelse{ \lengthtest{ \paperwidth = 297mm}}
{\geometry{top=1cm,left=1cm,right=1cm,bottom=1cm} }
{\geometry{top=1cm,left=1cm,right=1cm,bottom=1cm} }
}
\setlength{\parindent}{0em}
\setlength{\parskip}{-0.25em}
\pagestyle{fancy}
\fancyhf{}
\renewcommand{\headrulewidth}{0pt}
\rhead{Pete Wilcox | CruzID: pcwilcox | Student ID: 1593715}

\begin{document}
\footnotesize
\begin{multicols}{4}
    \setlength{\premulticols}{1pt}
    \setlength{\postmulticols}{1pt}
    \setlength{\multicolsep}{1pt}
    \setlength{\columnsep}{2pt}
    \begin{itemize}
        \item \code{ACID}
              \begin{itemize}
                  \item \code{ATOMICITY}: An atomic transaction happens as one unit, either the whole thing
                        commits or none of it does.
                  \item \code{CONSISTENCY}: A consistent transaction brings the DB from one valid state to
                        another valid state with respect to any constraints.
                  \item \code{ISOLATION}: Concurrent isolated transactions would have the same result if run sequentially.
                  \item \code{DURABILITY}: A committed transaction will remain committed even in the event of
                        a hardware failure.
              \end{itemize}
        \item \code{RAID Levels}
              \begin{itemize}
                  \item Level 0: No redundancy (just stripin)
                  \item Level 1: Mirrored (two identical copies)
                        \begin{itemize}
                            \item Each disk has an exact mirror image
                            \item Parallel reads; writes involve two disks
                            \item Maximum transfer rate = transfer rate of one disk
                        \end{itemize}
                  \item Level 0+1 (Level 10): Striping and Mirroring
                        \begin{itemize}
                            \item Parallel reads; writes involve two disks
                            \item Maximum transfer rate = aggregate bandwidth
                        \end{itemize}
                  \item Level 3: Bit-interleaved parity
                        \begin{itemize}
                            \item Striping Unit: one bit (or byte) (one check disk)
                            \item Each read and write request involves all disks; disk array can process one request at a time
                        \end{itemize}
                  \item Level 4: Block-interleaved parity
                        \begin{itemize}
                            \item Striping unit: one disk block (one check disk)
                            \item Parallel reads possible for small requests, large requests can utilize full bandwidth
                            \item Writes involve modified block \emph{and} check disk
                        \end{itemize}
                  \item Level 5: Block-interleaved distributed parity
                        \begin{itemize}
                            \item Similar to RAID level 4 but parity blocks are distributed over all disks
                        \end{itemize}
              \end{itemize}
        \item \code{Buffer Management in a DBMS}
              \begin{itemize}
                  \item DBMS maintains buffer pool of frames, each frame holds a page, info is in \code{<frame#, pageid>} table
                  \item Choice of frame replacement dictated by replacement policy such as LRU
                  \item When a page is requested:
                        \begin{itemize}
                            \item If requested page is not in pool:
                                  \begin{itemize}
                                      \item Choose a frame for replacement
                                      \item If that frame is dirty, write it to disk
                                      \item Read requested page into chosen frame
                                  \end{itemize}
                            \item Pin the page and return its address
                            \item When done the requestor must indicate whether the page has been modified (dirty bit) and unpin
                            \item Page in pool may be requested many times
                                  \begin{itemize}
                                      \item A pin count is used and a page is a candidate for replacement iff \code{pin_count = 0}
                                      \item Pinning increments pin count and unpinning decrements
                                  \end{itemize}
                            \item Concurrency control and recovery may entail additional I/O when a frame is chosen for replacement (write-ahead log protocol)
                            \item Frame is chosen for replacement using LRU, clock, MRU, etc
                            \item Sequential flooding: Caused by using LRU when the number of buffer frames is less than the number of pages in the file
                        \end{itemize}
              \end{itemize}
        \item \code{Files of Records}
              \begin{itemize}
                  \item Page or block is ok when doing I/O but higher levels of DBMS operate on \emph{records} and thus want \emph{files of records}
                  \item \code{FILE:} A collection of pages each containing a collection of records. Must support
                        \begin{itemize}
                            \item Insert (append)/delete/modify record
                            \item Read a particular record specified using \emph{record id}
                            \item Scan all records possibly with some conditions on the records to be retrieved
                        \end{itemize}
                  \item \code{Unordered ``Heap'' Files:}
                        \begin{itemize}
                            \item Simplest file structure that contains records in no particular (logical) order
                            \item As file grows and shrinks, disk pages are allocated and de-allocated
                            \item To support record-level operations we must:
                                  \begin{itemize}
                                      \item Keep track of the \emph{pages} in a file: \code{page id (pid)}
                                      \item Keep track of the \emph{free space} on a page
                                      \item Keep track of the \emph{records} on a page: \code{record id (rid)}
                                      \item Keep track of \emph{fields} within records
                                  \end{itemize}
                            \item Operations: create/destroy file, insert/delete record, fetch record with specific \code{rid}, scan all records
                        \end{itemize}
                  \item Record formats: \code{Fixed Length}
                        \begin{itemize}
                            \item Information about field types is the same for all records in file; it is stored in \emph{system catalogs}
                            \item Finding the $i^{th}$ field of a record does not require scanning the record
                        \end{itemize}
                  \item Record formats: \code{Variable length}
                        \begin{itemize}
                            \item Several alternative formats (\# of fields is fixed)
                            \item Fields delimited by special symbols (e.g. \$ between fields)
                            \item Fields preceded by lengths
                        \end{itemize}
                  \item  Record formats: \code{Variable length with directory}
                        \begin{itemize}
                            \item Use array of offsets at start of record
                        \end{itemize}
                  \item \code{Heap file implemented as a list}
                        \begin{itemize}
                            \item The header page id and heap file name must be stored someplace
                            \item Each page contains two extra pointers in this case
                            \item Refinement: use several lists for different degrees of free space
                        \end{itemize}
                  \item Page formats:
                        \begin{itemize}
                            \item File -> collection of pages
                            \item Page -> collection of tuples/records
                            \item Query operators deal with tuples
                            \item Slotted page format:
                                  \begin{itemize}
                                      \item Each page has a collection of \emph{slots}
                                      \item Each slot contains a record
                                  \end{itemize}
                            \item RID: \code{<page id, slot number>}
                        \end{itemize}
                  \item \code{Heap file using a page directory}
                        \begin{itemize}
                            \item Page entries can include the number of free bytes on each page
                            \item Directory is a collection of pages; linked list is one possible implementation
                        \end{itemize}
                  \item \code{System catalogs:}
                        \begin{itemize}
                            \item For each relation:
                                  \begin{itemize}
                                      \item name, file, file structure
                                      \item name, type, length (if fixed) for each attribute
                                      \item Index name, target, and kind for each index
                                      \item also integrity constraints, defaults, nullability, etc
                                  \end{itemize}
                            \item For each index: structure (e.g. B+ tree) and search key fields
                            \item For each view: view name and definition (including query)
                            \item Plus statistics, authorization, buffer pool size, etc
                        \end{itemize}
                  \item \code{Column Stores:}
                        \begin{itemize}
                            \item Store data ``vertically''
                            \item Contrast with a ``row-store'' that stores all the attributes of a tuple/record contiguously
                            \item Each column can be stored as a separate file and compressed
                            \item SAP HANA:
                                  \begin{itemize}
                                      \item Dictionary compression per column
                                      \item Column main: read-optimized store for immutable data. Uses high data compression and heuristic algoriths to order data to maximize secondary compression
                                      \item Column delta: write-optimized store for inserts, updates, deletes. Uses less compression, appends updates to the end, and merges with main periodically.
                                  \end{itemize}
                            \item Additional types: prefix coding, run length coding, cluster coding, sparse coding, indirect coding
                        \end{itemize}
                  \item \code{Indexes:}
                        \begin{itemize}
                            \item Speeds up selections on the search key fields for the index
                            \item Contains a collection of data entries and supports efficient retrieval of all data entires $k^*$ with a given key value $k$
                        \end{itemize}
                  \item \code{B+ Tree Indexes}
                        \begin{itemize}
                            \item Leaf pages contain \emph{data entries} and are chained (prev \& next)
                            \item Non-leaf pages have \emph{index entries}, used to direct searches
                            \item Insert/delete at $\log_F N$, keep tree \emph{height-balanced} ($F = $ fanout, $N = $ \# leaf pages)
                            \item Minimum 50\% occupancy (in all nodes except root). Each node contains $d \leq m \leq 2d$ entries; $d = $ the \emph{order} of the tree.
                            \item Typical order $d = 100$
                            \item Percentage of node that is full is more useful, typical fill-factor 67\%
                            \item Average \emph{fanout} for non-leaves $F = 133$
                            \item Inserting a data entry:
                                  \begin{itemize}
                                      \item Find correct leaf $L$
                                      \item Put data entry onto $L$
                                      \item If $L$ has enough space, done
                                      \item Otherwise, must split $L$. Redistribute entries evenly, copy up the middle key (key must still exist in leaf). Insert index entry pointing to $L_2$ into parent of $L$.
                                      \item This can happen recursively: if parent of $L$ grows, need to push up middle key.
                                      \item Splits ``grow'' the tree; root split increases height.
                                  \end{itemize}
                            \item Deleting a data entry:
                                  \begin{itemize}
                                      \item Start at root, find leaf $L$ where entry belongs
                                      \item Remove the entry
                                      \item If $L$ is at least half full, done
                                      \item Otherwise, if $L$ has only $d-1$ entries, try to redistribute, borrowing from sibling (adjacent node with same parent)
                                      \item If redistribution fails, merge $L$ and sibling
                                      \item If merge occurred, must delete entry from parent (pointing to merged node)
                                      \item Merge can propagate to root, decreasing height of the tree
                                  \end{itemize}
                        \end{itemize}
              \end{itemize}
    \end{itemize}
    \begin{center}
        \begin{tabular}{ | c | c | c | c | c | } \hline
            $p$ & $q$ & $p$ OR $q$ & $p$ AND $q$ & $p = q$ \\ \hline
            T   & T   & T          & T           & T       \\ \hline
            T   & F   & T          & F           & F       \\ \hline
            T   & U   & T          & U           & U       \\ \hline
            F   & T   & T          & F           & F       \\ \hline
            F   & F   & F          & F           & T       \\ \hline
            F   & U   & U          & F           & U       \\ \hline
            U   & T   & T          & U           & U       \\ \hline
            U   & F   & U          & F           & U       \\ \hline
            U   & U   & U          & U           & U       \\ \hline
        \end{tabular}
    \end{center}
    \begin{center}
        \begin{tabular}{| c | c |} \hline
            $p$ & NOT $p$ \\ \hline
            T   & F       \\ \hline
            F   & T       \\ \hline
            U   & U       \\ \hline
        \end{tabular}
    \end{center}
    \begin{center}
        \begin{adjustbox}{angle=270}
            \begin{tabular}{| c | c | c | c |} \hline
                Isolation Level  & dirty reads & non-repeatable reads & phantoms \\ \hline
                READ UNCOMMITTED & Y           & Y                    & Y        \\ \hline
                READ COMMITTED   & N           & Y                    & Y        \\ \hline
                REPEATABLE READ  & N           & N                    & Y        \\ \hline
                SERIALIZABLE     & N           & N                    & N        \\ \hline
            \end{tabular}
        \end{adjustbox}
    \end{center}
\end{multicols}

\end{document}
